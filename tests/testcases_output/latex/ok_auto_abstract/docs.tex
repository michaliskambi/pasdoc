\documentclass{report}
\usepackage{hyperref}
% WARNING: THIS SHOULD BE MODIFIED DEPENDING ON THE LETTER/A4 SIZE
\oddsidemargin 0cm
\evensidemargin 0cm
\marginparsep 0cm
\marginparwidth 0cm
\parindent 0cm
\setlength{\textwidth}{\paperwidth}
\addtolength{\textwidth}{-2in}


% Conditional define to determine if pdf output is used
\newif\ifpdf
\ifx\pdfoutput\undefined
\pdffalse
\else
\pdfoutput=1
\pdftrue
\fi

\ifpdf
  \usepackage[pdftex]{graphicx}
\else
  \usepackage[dvips]{graphicx}
\fi

% Write Document information for pdflatex/pdftex
\ifpdf
\pdfinfo{
 /Author     (Pasdoc)
 /Title      ()
}
\fi


\begin{document}
\label{toc}\tableofcontents
\newpage
% special variable used for calculating some widths.
\newlength{\tmplength}
\chapter{Unit ok{\_}auto{\_}abstract}
\label{ok_auto_abstract}
\index{ok{\_}auto{\_}abstract}
\section{Description}
This is the 1st sentence, it will be turned into @abstact description of this item. This is the 2nd sentence of the description.
\section{Overview}
\begin{description}
\item[\texttt{\begin{ttfamily}TTest1\end{ttfamily} Class}]This is the explicit abstract section
\item[\texttt{\begin{ttfamily}TTest2\end{ttfamily} Class}]In this case there is no period char '.' that is followed by whitespace in this comment, so the whole comment will be treated as abstract description
\item[\texttt{\begin{ttfamily}TTest3\end{ttfamily} Class}]Of course, 1st sentence may contain other tags, like this: \begin{ttfamily}TTest1\end{ttfamily}(\ref{ok_auto_abstract.TTest1}) and like this: \begin{ttfamily}Some code. Not really Pascal code, but oh well...\end{ttfamily} and I'm still in the 1st sentence, here the @abstract part ends.
\item[\texttt{\begin{ttfamily}TTest4\end{ttfamily} Class}]First sentence, auto{-}abstracted, and the 1st paragraph at the same time.
\end{description}
\section{Classes, Interfaces, Objects and Records}
\ifpdf
\subsection*{\large{\textbf{TTest1 Class}}\normalsize\hspace{1ex}\hrulefill}
\else
\subsection*{TTest1 Class}
\fi
\label{ok_auto_abstract.TTest1}
\index{TTest1}
\subsubsection*{\large{\textbf{Hierarchy}}\normalsize\hspace{1ex}\hfill}
TTest1 {$>$} TObject
\subsubsection*{\large{\textbf{Description}}\normalsize\hspace{1ex}\hfill}
This is the explicit abstract section\hfill\vspace*{1ex}

This is the 1st sentence of description.

This is the 2nd sentence of description.

\ifpdf
\subsection*{\large{\textbf{TTest2 Class}}\normalsize\hspace{1ex}\hrulefill}
\else
\subsection*{TTest2 Class}
\fi
\label{ok_auto_abstract.TTest2}
\index{TTest2}
\subsubsection*{\large{\textbf{Hierarchy}}\normalsize\hspace{1ex}\hfill}
TTest2 {$>$} TObject
\subsubsection*{\large{\textbf{Description}}\normalsize\hspace{1ex}\hfill}
In this case there is no period char '.' that is followed by whitespace in this comment, so the whole comment will be treated as abstract description\ifpdf
\subsection*{\large{\textbf{TTest3 Class}}\normalsize\hspace{1ex}\hrulefill}
\else
\subsection*{TTest3 Class}
\fi
\label{ok_auto_abstract.TTest3}
\index{TTest3}
\subsubsection*{\large{\textbf{Hierarchy}}\normalsize\hspace{1ex}\hfill}
TTest3 {$>$} TObject
\subsubsection*{\large{\textbf{Description}}\normalsize\hspace{1ex}\hfill}
Of course, 1st sentence may contain other tags, like this: \begin{ttfamily}TTest1\end{ttfamily}(\ref{ok_auto_abstract.TTest1}) and like this: \begin{ttfamily}Some code. Not really Pascal code, but oh well...\end{ttfamily} and I'm still in the 1st sentence, here the @abstract part ends. This is the 2nd sentence.

Note that in this example the '.' char inside @code tag did not confuse pasdoc -- it was not treated as the end of 1st sentence, because it was part of parameters of @code tag. Even though @code tag in the example above used special syntax TagsParametersWithoutParenthesis.\ifpdf
\subsection*{\large{\textbf{TTest4 Class}}\normalsize\hspace{1ex}\hrulefill}
\else
\subsection*{TTest4 Class}
\fi
\label{ok_auto_abstract.TTest4}
\index{TTest4}
\subsubsection*{\large{\textbf{Hierarchy}}\normalsize\hspace{1ex}\hfill}
TTest4 {$>$} TObject
\subsubsection*{\large{\textbf{Description}}\normalsize\hspace{1ex}\hfill}
First sentence, auto{-}abstracted, and the 1st paragraph at the same time.

Notice that html output will add {$<$}p{$>$} to DetailedDescription, but not to AbstractDescription. This is second paragraph.\end{document}
